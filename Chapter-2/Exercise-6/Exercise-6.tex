Utilizzare la $function$ del precedente esercizio per determinare un'approssimazione della
radice della funzione
$$f(x)=x-e^{-x}cos(\frac{x}{100})$$
per $tol=10^{-i}, i=1,2,...,12$, partendo da $x_\mathrm{0}=-1$. Per il metodo di bisezione utilizzare
$[-1,1]$, come intervallo di confidenza iniziale. Tabulare i risultati, in modo da confrontare le iterazioni
richieste da ciascun metodo. Commentare il relativo costo computazionale.

\hspace*{\fill}
\par\noindent\rule{\textwidth}{0.4pt}
\hspace*{\fill}

\begin{lstlisting}[language=Matlab, caption=Codice Matlab]
f = @(x) x - exp(-x) * cos(x / 100);
f1 = @(x) 1 + exp(-x) * cos(x / 100) + (exp(-x) * sin(x / 100)) / 100;
x0 = -1;
a = -1;
b = 1;
imax = 1000;
for i = 1 : 12
	tol = 10^(-i);
	rtol=['Tolleranza: ',num2str(tol, '%e')];
	disp(rtol)
	[xB, it] = bisection(f, a, b, tol);
	disp(['Bisezione: ', num2str(xB, '%10.12e'), ' Iterazioni ', num2str(it)])
	[xN, it] = newton(f, f1, x0, imax, tol);
	disp(['Newton: ', num2str(xN, '%10.12e'), ' Iterazioni ', num2str(it)])
	[xC, it] = chord(f, f1, x0, imax, tol);
	disp(['Corde: ', num2str(xC, '%10.12e'), ' Iterazioni ', num2str(it)])
	[xS, it] = secant(f, f1, x0, imax, tol);
	disp(['Secanti: ', num2str(xS, '%10.12e'), ' Iterazioni ', num2str(it)])
	disp(' ')
end
\end{lstlisting}

\begin{table}[H]
\caption{Risultati}
\centering
\resizebox{\textwidth}{!}{%
\begin{tabular}{|c|c|c|c|c|}
\hline
& \multicolumn{2}{c|}{\textbf{Bisezione}} & \multicolumn{2}{c|}{\textbf{Newton}} \\ \hline
Tolleranza & Risultato 	& Iterazioni  & Risultato & Iterazioni \\ \hline
$10^{-1}$  & 5.000000000000e-01 & 3  & 5.663058026183e-01 & 2  \\ \hline
$10^{-2}$  & 5.625000000000e-01 & 6  & 5.671373451066e-01 & 3  \\ \hline
$10^{-3}$  & 5.664062500000e-01 & 10 & 5.671373451066e-01 & 3  \\ \hline
$10^{-4}$  & 5.671386718750e-01 & 14 & 5.671374702932e-01 & 4  \\ \hline
$10^{-5}$  & 5.671386718750e-01 & 14 & 5.671374702932e-01 & 4  \\ \hline
$10^{-6}$  & 5.671386718750e-01 & 14 & 5.671374702932e-01 & 4  \\ \hline
$10^{-7}$  & 5.671374797821e-01 & 24 & 5.671374702932e-01 & 4  \\ \hline
$10^{-8}$  & 5.671374797821e-01 & 24 & 5.671374702932e-01 & 5  \\ \hline
$10^{-9}$  & 5.671374704689e-01 & 31 & 5.671374702932e-01 & 5  \\ \hline
$10^{-10}$ & 5.671374702360e-01 & 34 & 5.671374702932e-01 & 5  \\ \hline
$10^{-11}$ & 5.671374702943e-01 & 36 & 5.671374702932e-01 & 5  \\ \hline
$10^{-12}$ & 5.671374702943e-01 & 36 & 5.671374702932e-01 & 5  \\ \hline
\end{tabular}%
}
\end{table}
\begin{table}[H]
	\centering
	\resizebox{\textwidth}{!}{%
	\begin{tabular}{|c|c|c|c|c|}
	\hline
	& \multicolumn{2}{c|}{\textbf{Secanti}} & \multicolumn{2}{c|}{\textbf{Corde}} \\ \hline
	Tolleranza &  Risultato & Iterazioni & Risultato & Iterazioni \\ \hline
	$10^{-1}$  & 5.662928457961e-01 & 3 & 4.021808606807e-01 & 3  \\ \hline
	$10^{-2}$  & 5.671339926161e-01 & 4 & 5.495185718942e-01 & 7  \\ \hline
	$10^{-3}$  & 5.671339926161e-01 & 4 & 5.651741531555e-01 & 11 \\ \hline
	$10^{-4}$  & 5.671374697616e-01 & 5 & 5.670103627779e-01 & 16 \\ \hline
	$10^{-5}$  & 5.671374697616e-01 & 5 & 5.671232371728e-01 & 20 \\ \hline
	$10^{-6}$  & 5.671374702932e-01 & 6 & 5.671358764609e-01 & 24 \\ \hline
	$10^{-7}$  & 5.671374702932e-01 & 6 & 5.671372918144e-01 & 28 \\ \hline
	$10^{-8}$  & 5.671374702932e-01 & 6 & 5.671374503070e-01 & 32 \\ \hline
	$10^{-9}$  & 5.671374702932e-01 & 6 & 5.671374689985e-01 & 37 \\ \hline
	$10^{-10}$ & 5.671374702932e-01 & 7 & 5.671374701482e-01 & 41 \\ \hline
	$10^{-11}$ & 5.671374702932e-01 & 7 & 5.671374702770e-01 & 45 \\ \hline
	$10^{-12}$ & 5.671374702932e-01 & 7 & 5.671374702914e-01 & 49 \\ \hline
	\end{tabular}%
	}
\end{table}
Analizzando i risultati ottenuti, abbiamo verificato che i metodi di bisezione e corde impiegano
un numero considerevolmente maggiore di iterazioni rispetto ai metodi di Newton e secanti, a paritá di tolleranza utilizzata.
Questo è dato dal fatto che i primi due hanno convergenza lineare, mentre i secondi due hanno convergenza quadratica.
Inoltre, studiando il costo computazionale per iterazione dei vari metodi, si ha che bisezione, secanti e corde richiedono una sola valutazione funzionale, mentre Newton ne richiede 2. L'unico metodo che non richiede la derivabilità della funzione è quello di bisezione, che assume infatti la sola continuità di $f(x)$.
Tuttavia i costi di iterazione più elevati e i maggiori requisiti sulla regolarità della funzione degli altri metodi, sono ripagati da più elevati ordini di convergenza.
