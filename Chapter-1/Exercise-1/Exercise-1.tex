Verificare che, per $h$ sufficientemente piccolo:
	\[
		\frac{3}{2} f(x) - 2f(x-h) + \frac{1}{2} f(x-2h)= hf'(x) + O(h^3)
	\]

\hspace{1cm}
\par\noindent\rule{\textwidth}{0.4pt}
\hspace{1cm}

Usando gli sviluppi di Taylor fino al secondo ordine otteniamo:
	\[
	f(x) = f(x_\mathrm{0}) + f'(x_\mathrm{0})h + \frac{1}{2} f''(x_\mathrm{0})h^{2} + O(h^3)
	\]
	\[
	f(x-h) = f(x_\mathrm{0}) + f'(x_\mathrm{0})(x-h-x_\mathrm{0}) + \frac{1}{2} f''(x_\mathrm{0})(x-h-x_\mathrm{0})^{2} + O(h^3)
	\]
	\[
	f(x-2h) = f(x_\mathrm{0}) + f'(x_\mathrm{0})(x-2h-x_\mathrm{0}) + \frac{1}{2} f''(x_\mathrm{0})(x-2h-x_\mathrm{0})^{2} + O(h^3)
	\]
Dato che $x-x_\mathrm{0}=h$, ponendo $x_\mathrm{0}=x-h$, si ha che:
	\[
	f(x-h)=f(x_\mathrm{0})+ O(h^3)
	\]
	\[
	f(x-2h) = f(x_\mathrm{0}) - f'(x_\mathrm{0})h + \frac{1}{2} f''(x_\mathrm{0})h^{2} + O(h^3)
	\]
Effettuando le opportune sostituzioni, la relazione iniziale diventa:
	\[
	\frac{3}{2}f(x_\mathrm{0}) + \frac{3}{2}f'(x_\mathrm{0})h + \frac{3}{4} f''(x_\mathrm{0})h^{2} - 2f(x_\mathrm{0}) + \frac{1}{2}f(x_\mathrm{0}) - \frac{1}{2}f'(x_\mathrm{0})h + \frac{1}{4} f''(x_\mathrm{0})h^{2} + O(h^3)=hf'(x) + O(h^3)
	\]
Semplificando, si ottiene che:
	\[
	f'(x_\mathrm{0})h+f''(x_\mathrm{0})h^{2}+O(h^3)=hf'(x) + O(h^3)
	\]
A questo punto, scriviamo lo sviluppo di Taylor anche per $f'(x)$:
	\[
	f'(x) = f'(x_\mathrm{0}) + f''(x_\mathrm{0})h + O(h^2)
	\]
Infine, otteniamo che:
	\[
	f'(x_\mathrm{0})h+f''(x_\mathrm{0})h^{2}+O(h^3)=f'(x_\mathrm{0})h + f''(x_\mathrm{0})h^{2} + O(h^3)
	\]
Dunque, l'uguaglianza iniziale è verificata.
