Scaricare la function \textbf{cremat} al sito
\url{http://web.math.unifi.it/users/brugnano/appoggio/cremat.m}
che crea sistemi lineari $n \times n$ la cui soluzione è il vettore $\textbf{x}=(1 \dots n)^{T}$.
Eseguire, quindi, lo $script$ Matlab:
\begin{lstlisting}[language=Matlab]
n = 10;
x = zeros(n, 15);
for i = 1 : 15
[A, b] = cremat(n, i);
[LU, p] = palu(A);
x(:, i) = lusolve(LU, p, b);
end
\end{lstlisting}
Confrontare i risultati ottenuti con quelli attesi, e dare una spiegazione esauriente degli stessi.

\hspace*{\fill}
\par\noindent\rule{\textwidth}{0.4pt}
\hspace*{\fill}

\textbf{Funzione cremat}:
\lstinputlisting[language=Matlab]{Chapter-3/Exercise-10/cremat.m}
