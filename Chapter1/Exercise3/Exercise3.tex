Eseguire il seguente $script$ Matlab:
\begin{lstlisting}[caption = {}]
	format long e
	n=75;
	u=1e-300;for i=1:n,u=u*2;end,for i=1:n,u=u/2;end,u
	u=1e-300;for i=1:n,u=u/2;end,for i=1:n,u=u*2;end,u
\end{lstlisting}
Spiegare i risultati ottenuti.

\hspace{1cm}
\par\noindent\rule{\textwidth}{0.4pt}
\hspace{1cm}

I risultati ottenuti:
\begin{lstlisting}[caption = {}]
u = 
	1.000000000000000e-300

u = 
	1.119916342203863e-300
\end{lstlisting}

L'obiettivo del programma proposto è verificare, dopo una serie di operazioni, che il numero restituito sia uguale a quello di partenza e
che quindi non si abbia perdita di informazione.

Prima di tutto viene impostato il formato di output a $longE$, formato long decimal (15 cifre dopo la virgola) con notazione scientifica.\\

Alcune premesse:\\
In caso di underflow, ovvero quando un calcolo produce un valore più piccolo di $realmin$, in accordo con l’opzione dello standard IEEE, 
in MATLAB i numeri vengono denormalizzati, perdono quindi il primo 1 sottinteso e sono definiti dalla loro rappresentazione binaria.
Inoltre, il più piccolo numero positivo denormalizzato è $\numprint{0.494}\times10^{-323}$. Ogni risultato più piccolo di questo è posto a zero.\\

Alcune notazioni importanti:
\begin{itemize}
	\item $realmin$ è il minimo numero normalizzato rappresentabile in MATLAB
	\item $realmax$ è il massimo numero normalizzato rappresentabile in MATLAB
	\item $eps$ è la precisione di macchina
\end{itemize}

I primi 2 for effettuano prima 75 moltiplicazioni per 2 del numero $u$ e poi lo stesso numero di divisioni per 2 di $u$, 
mantenendone il valore al di sopra di $realmin$ e facendo sì che non ci sia perdita di informazione (underflow).
I for successivi invece, a causa dello svolgersi prima delle 75 divisioni e poi delle 75 moltiplicazioni, portano alla denormalizzazione del numero e
quindi a una sua approssimazione, che risulta nel valore finale di $u$.
