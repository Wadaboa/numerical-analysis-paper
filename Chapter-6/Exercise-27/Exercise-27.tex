Scrivere le function ausiliarie, per la function del precedente esercizio, che implementano i metodi
iterativi di Jacobi e Gauss-Seidel.

\hspace*{\fill}
\par\noindent\rule{\textwidth}{0.4pt}
\hspace*{\fill}

\begin{minipage}{\textwidth}
\textbf{Metodo di Jacobi per la funzione \textit{itersolve}}:
    \lstinputlisting[language=Matlab]{Chapter-6/Exercise-27/iterjacobi.m}
    \textbf{Metodo di Gauss-Seidel per la funzione \textit{itersolve}}:
    \lstinputlisting[language=Matlab]{Chapter-6/Exercise-27/itergs.m}
\end{minipage}
\begin{minipage}{\textwidth}
    \textbf{Metodo di Jacobi per la funzione \textit{gsplit}}:
    \lstinputlisting[language=Matlab]{Chapter-6/Exercise-27/jacobi.m}
    \textbf{Metodo di Gauss-Seidel per la funzione \textit{gsplit}}:
    \lstinputlisting[language=Matlab]{Chapter-6/Exercise-27/gs.m}
\end{minipage}
