Scrivere le function ausiliarie, per la function del precedente esercizio, che implementano i metodi
iterativi di Jacobi e Gauss-Seidel.

\hspace*{\fill}
\par\noindent\rule{\textwidth}{0.4pt}
\hspace*{\fill}

\textbf{Metodo di Jacobi}:
\lstinputlisting[language=Matlab]{Chapter-6/Exercise-27/jacobi.m}
\textbf{Metodo di Gauss-Seidel}:
\lstinputlisting[language=Matlab]{Chapter-6/Exercise-27/gauss.m}
